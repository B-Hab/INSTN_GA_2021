\pdfminorversion=4
\documentclass[usenames,dvipsnames]{beamer}
%Pour impression des transparents sans animations
%option trans (à préciser aussi dans les \only
%\documentclass[trans]{beamer}
%
%Pour impression des transparents sur papier (pour les CRR)
%\documentclass[handout]{beamer}
%

\usetheme{cea2019}

\setbeamerfont{structure}{size=\small}
\setbeamerfont{title page}{size=\Large,parent=structure}
\setbeamerfont{frametitle}{size=\large,series=\bfseries,parent=structure}
\setbeamerfont{headline}{size=\small}
\setbeamertemplate{bibliography item}[mybibitem]
\setbeamerfont{bibliography entry author}{shape=\upshape,size=\tiny}%
\setbeamerfont{bibliography entry title}{shape=\upshape,size=\tiny}
\setbeamerfont{bibliography entry journal}{shape=\upshape,size=\tiny}
\setbeamerfont{bibliography entry note}{shape=\upshape,size=\tiny}
\setbeamerfont{itemize/enumerate body}{size=\footnotesize}
\setbeamerfont{itemize/enumerate subbody}{size=\footnotesize}
\setbeamerfont{itemize/enumerate subsubbody}{size=\scriptsize}
\setbeamerfont{block body}{size=\footnotesize,parent={structure}}
\setbeamerfont{block body alerted}{parent={block body}}
\setbeamerfont{block body example}{parent={block body}}
\setbeamerfont{block title}{size=\normalsize,series=\bfseries,parent={block body}}
\setbeamerfont{section in head/foot}{size=\scriptsize,shape=\scshape,series=\bfseries}
\setbeamerfont{subsection in head/foot}{shape=\upshape,parent={section in head/foot}}
\setbeamerfont{section in toc}{parent=structure,series=\bfseries}
\setbeamerfont{caption}{size=\scriptsize, parent=structure}

\setbeamertemplate{caption}{\centering\insertcaption\par}
\renewcommand{\emph}[1]{\textcolor{cea8}{\textit{#1}}}
\bibliographystyle{apalike}


\usepackage{tabularx}
\newcommand{\n}{\tabularnewline}
\newcolumntype{C}{>{\centering}X}
\newcolumntype{R}{>{\raggedleft}X} 
\newcolumntype{L}{>{\raggedright}X} 
\newcolumntype{M}[1]{>{\centering}m{#1}}
\usepackage{multirow}
\newenvironment{legend}%
{\tabular{r>{\small} l}}%
{\endtabular}
\usepackage{hyperref}
\usepackage{cancel}
\usepackage{tikzsymbols}
\usepackage{mhchem}
\usepackage{wasysym}
\usepackage{animate}
\usepackage{multimedia}
\usepackage{fourier-orns}
\usepackage{dirtree}

\newcommand{\grad}[1]{\nabla #1}
\renewcommand{\div}[1]{\nabla \cdot #1}
\newcommand{\Lapl}[1]{\Delta #1}

\newenvironment{remark}[1][\textit{Nota Bene}]{\begin{trivlist}
\item[\hskip \labelsep {\bfseries \rule{1ex}{1ex} #1}]\ignorespaces}{\rule{1ex}{1ex} \end{trivlist}\ignorespacesafterend}

%\usepackage[default,scale=0.85]{opensans} 
%\usepackage{lmodern} 
%
\titre{\normalsize{Accidents graves des réacteurs nucléaires} \\ \Large{Comportement et rétention du corium en cuve d'un réacteur à eau légère}}
\evenement{INSTN - GA - $\mu$-projet}
\auteurs{Romain Le Tellier, Louis Viot, Benoît Habert, CEA Cadarache \\ {\scriptsize \href{mailto:romain.le-tellier@cea.fr}{romain.le-tellier@cea.fr}, \href{mailto:louis.louis@cea.fr}{louis.louis@cea.fr}, \href{mailto:benoit.habert.fr}{benoit.habert@cea.fr}}}
\datedocument{Mars 2020}
\auteurprincipal{INSTN - GA - $\mu$-projet}
%%
\begin{document}

%% TITLE PAGE %%
\PageTitre{}
%%

%%%%%%%%
\section{Objectifs pédagogiques}
\Titre{Objectifs pédagogiques}
\begin{frame}[fragile]
\begin{itemize}
\item Connaître les \emph{phénomènes physiques} déterminant vis-à-vis du comportement du \emph{corium dans le fond de cuve} et le niveau de connaissance associé
\item Faire le lien entre ces phénomènes et un savoir de base en \emph{thermohydraulique, thermodynamique}
\item Comprendre les \emph{modélisations} mises en \oe uvre dans les codes intégraux pour l'\emph{évaluation du risque de percement de la cuve} d'un réacteur à eau légère dans une \emph{stratégie de rétention du corium en cuve}
\item Réaliser une telle \emph{évaluation}, \emph{en stationnaire}, puis \emph{en transitoire} avec, finalement, une évaluation des \emph{sensibilités à certains paramètres de modèles}
\item Avoir une idée concrète des \emph{activités de R\&D menées au CEA} sur ce sujet
\end{itemize}
\end{frame}
%%%%%%%%
\section{Déroulement du $\mu$-projet}
\Titre{Déroulement du $\mu$-projet}
\begin{frame}[fragile]
\begin{itemize}
\item \emph{1$^{\text{ère}}$ séance ``encadrée''} : introduction à la stratégie de rétention du corium en cuve, phénoménologie d'un bain de corium ``à deux couches'' en configuration stationnaire, démarrage du TD associé
\item \emph{1$^{\text{ère}}$ séance ``libre''} : réalisation du TD, rédaction
\item \emph{2$^{\text{ème}}$ séance ``encadrée''} : discussion sur le TD, phénoménologie d'un bain de corium en transitoire, introduction au code PROCOR, premier calcul avec PROCOR
\item \emph{3$^{\text{ème}}$ séance ``encadrée''} : suite des calculs avec PROCOR, introduction au calcul statistique avec PROCOR/URANIE
\item \emph{2$^{\text{ème}}$ séance ``libre''} : analyse d'un calcul statistique avec PROCOR/URANIE, rédaction
\item \emph{Rédaction}
\item \emph{Soutenance}
\end{itemize}
\end{frame}
%%

%%%%%%%%
%\setbeamertemplate{background canvas}[annexe]
\section*{Sommaire}
%\Titre{Sommaire}
\begin{frame}[fragile]
\frametitle{Sommaire}
  \linespread{0.85}
  \tableofcontents[sectionstyle=hide/show, subsectionstyle=hide/show/show,sections={3-}, firstsection=3]
  \linespread{1}
\end{frame}
%\setbeamertemplate{background canvas}{}

%%%%%%%%%%%%%%%%%%%%%%%%%%%%%%%%%%%%%%%%%%%%%%%%%%%%%%%%%%%%%%%%%%%%%%%%%%%%%%%%%%%%%%%%%%%%%%%%%%%%%%%%%%%%%%%%%%%%%%%%%%%%%%%%%%%%%%%%%%%%%%%%%%%%%%%%%
\input{part1.tex}

%%%%%%%%%%%%%%%%%%%%%%%%%%%%%%%%%%%%%%%%%%%%%%%%%%%%%%%%%%%%%%%%%%%%%%%%%%%%%%%%%%%%%%%%%%%%%%%%%%%%%%%%%%%%%%%%%%%%%%%%%%%%%%%%%%%%%%%%%%%%%%%%%%%%%%%%%
\input{part2.tex}

%%%%%%%%%%%%%%%%%%%%%%%%%%%%%%%%%%%%%%%%%%%%%%%%%%%%%%%%%%%%%%%%%%%%%%%%%%%%%%%%%%%%%%%%%%%%%%%%%%%%%%%%%%%%%%%%%%%%%%%%%%%%%%%%%%%%%%%%%%%%%%%%%%%%%%%%%

%%%%%%%%
\Intercalaire{Introduction au calcul statistique avec PROCOR/URANIE}
\section{Introduction au calcul statistique avec PROCOR/URANIE}
\Titre{Introduction au calcul statistique avec PROCOR/URANIE}
\begin{frame}[fragile]
\end{frame}

%%%%%%%%%%%%%%%%%%%%%%%%%%%%%%%%%%%%%%%%%%%%%%%%%%%%%%%%%%%%%%%%%%%%%%%%%%%%%%%%%%%%%%%%%%%%%%%%%%%%%%%%%%%%%%%%%%%%%%%%%%%%%%%%%%%%%%%%%%%%%%%%%%%%%%%%%
\DernierePage{}
\Annexes
\section*{Références}
\Titre{Références}
\begin{frame}[fragile,allowframebreaks]
  \bibliography{lma-jabref}
\end{frame}
%%
\end{document}
%% definition du dictionnaire local
<!-- Local IspellDict: francais -->
